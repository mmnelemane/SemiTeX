%===============================================================================
% Zweck:    KTR-Seminar-Vorlage
% Erstellt: 16.10.2007
% Updated:  27.06.2016
% Autor:    U.K. / M.G.
%===============================================================================
\RequirePackage[hyphens]{url}
\documentclass[journal, onecolumn, a4paper, 12pt]{IEEEtran}
%===============================================================================
% zentrale Layout-Angaben und Befehle
%===============================================================================
\newcommand\meta{./meta}
\newcommand\images{./images}
\newcommand\content{./content}

\input{\meta/config/commands}

%===============================================================================
% LATEX-Dokument
%===============================================================================
\input{\meta/config/hyphenation}
\begin{document}
  %===============================================================================
  % Zum Kompilieren pdflatex und bibtex ausführen.
  % Konfiguration in texmaker: Options -> Configure Texmaker -> Quick Build -> Select Latexmk + ViewPDF
  % Entsprechende Informationen in den config/metainfo verändern
  % Zur Auswahl der Sprache im folgenden Befehl
  % ngerman für deutsch eintragen, english für Englisch.
  %===============================================================================
\selectlanguage{english}

\maketitle

\pagenumbering{Roman}
\setcounter{page}{2}
%
\tableofcontents
% Einstellungen f\"{u}r Literaturverzeichnis
\newpage
\addcontentsline{toc}{section}{\listfigurename}
\listoffigures
\newpage
\addcontentsline{toc}{section}{\listtablename}
\listoftables
\newpage
\printnomenclature
%===============================================================================
% LATEX-Dokument: Kapitel laden
%===============================================================================
%
\newpage
\pagenumbering{arabic}
\setcounter{page}{1}
%
% to use git tagging
%
%\ifgit
%  \input{\meta/exampleContent/version}
%\fi
%
% hier einzelne Kapitel mit \input{Kapitel-File} einf\"{u}gen
%
\input{./content/introduction}
\input{./content/architecture_overview}
\subsection{MEC Architecture}

ETSI establishes a reference framework and architecture to realize MEC as shown below:

\begin{figure}[h!]
    \centering
    \includegraphics[width=0.9\textwidth]{mec_ref_architecture}
    \label{fig:4}
    \caption{ETSI MEC Reference Architecture. \protect\cite{etsimec03}}
\end{figure}

The architecture comprises the following 4 components \cite{etsimec03}:
\begin{enumerate}
    \item \textit{The MEC host} – provides the necessary infrastructure including virtualized resources.
    \item \textit{The MEC Platform} – Provides functionality and interfaces to run applications on top of the MEC host.
    \item \textit{MEC Management} – Handles host and system level management. It can be done at two levels – the host level involving the platform manager and the infrastructure manager and the system level using the MEC Edge orchestrator.
    \item \textit{MEC Applications} – Are services provided through the MEC architecture some of which can also be provided by the MEC platform manager.
\end{enumerate}

\subsection{NFV Architecture}

The goal of NFV  is to be able to run network services traditionally hosted in specialized hardware in virtualized resources like virtual machines and containers. This gives the necessary flexibility for realizing use-cases discussed above.  The following figure shows the reference architecture of a typical NFV deployment. The essential framework is outlined by etsi at \cite{etsi02}.

\begin{figure}[h!]
	\centering
	\includegraphics[width=0.9\textwidth]{nfv_ref_architecture}
	\label{fig:figure5}
    \caption{ETSI NFV Reference Architecture \cite{etsi02}}
\end{figure}

The NFV Architecture Framework in the above figure is a reference for designing an NFV solution and proposes the following components:

\begin{enumerate}
    \item \textit{Virtual Network Function}
        The functionality that traditionally runs on dedicated hardware and is intended to be virtualized. The functions could be the basic Enterprise functions like Firewall, DNS etc or specialized telecommunication functions like packet processing, mobility management etc,. The VNF can run on a single entity like a virtual machine or a container or on a cluster of virtual machines or containers. The functions that use containers as compute resources are sometimes referred to as Containerized Network Functions (CNF). 

    \item \textit{Element Management}
        The Element Management unit can be used to manage single of multipl VNF units for combined functionality. 

    \item \textit{NFV Infrastructure}
        NFVI refers to the virtualization infrastructure resources providing the compute, network and storage requirements of the VNFs. The infrastructure could comprise of Virtual Machines or Containers, Block and Object Storage units, Network artifacts and the hardware resources that hosts these resources.   

    \item \textit{Virtualized Infrastructure Manager}
        The VIM provides the management functions for the virtualized infrastructure. The functions include inventory management, resource allocation, monitoring, capacity planning, fault management, resource optimization and policy management.

    \item \textit{NFV Orchestrator}
        The Orchestrator is responsible for translating the requirements of a VNF into corresponding resources on the Virtualized Infrastructure. A standard format of description can be used for describing the requirements that can be answered by different VIM solutions.

    \item \textit{VNF Manager}
        The VNF Manager manages the lifecycle of the VNF including instantiation, update, query, scaling, termination and service availability

    \item \textit{Service, VNF and Infrastructure Description}
        This describes a model to describe the entities in the NFV architecture including the Service, VNF and the Infrastructure. The model should be standard and universal so that the same description can be used to abstractly realize multiple types of infrastructure solutions. 

    \item \textit{Operations and Business Support Systems - OSS/BSS}
        The Orchestration and Management blocks of the NFV reference architecture are expected to expose standard APIs for applications. The set of applications that monitor and collect various measurement data from the NFV system can also provide for operational and business functions like service provisioning, billing, auditing etc,. that are essential functions for an operator in order to monetize the solution offered through NFV.

\end{enumerate}

\subsection{Combined Architecture for NFV an MEC}

\begin{figure}[ht!]
    \centering
    \includegraphics[width=0.9\textwidth]{images/combined_architecture}
    \label{fig:figure6}
    \caption{A Combined architecture for NFV and MEC}
\end{figure}

The NFV Infrastructure provides the necessary foundation for realizing all the pillars including MEC and NFV\@. The infrastructure mainly provides functions to automatically deploy and manage virtualized resources. Several requirements for NFV and MEC overlap and due to the similarities, a combined architecture for NFV and MEC was proposed in \cite{taleb17}. The idea is to use a common Infrastructure platform to run Virtual Application Functions (VAFs) and Virtual Network Functions (VNFs). The corresponding orchestration modules are named as AFVO and NFVO\@. This common platform allows for reduced design issues and massive reuse of components thereby saving in CapeX and OpeX costs. A paper in [2] represents a combined architecture as in the above diagram.


Here is a table comaparing the terminologies used between NFV and MEC

\begin{tabular}{||p{4cm}|p{4cm}|p{4cm}||}
\hline\hline
Component&NFV&MEC \\
\hline
Managed Entity&VNF&VAF \\
\hline
Orchestrator&NFVO&AFVO \\
\hline
Platform Manager&VNF Manager&VAF Manager \\
\hline
Infrastructure&NFV Infrastructure&Mobile Edge Host \\
\hline\hline
\end{tabular}

\input{./content/requirements}
\section{Enabling Technologies}

A range of technologies developed over several years have enabled the evolution of NFV and Edge Computing. These technologies can be grouped among four planes of computing. This model is applicable to MEC as well as the related technology called NFV.

\begin{figure}[h!]
	\centering
    \includegraphics[width=0.7\textwidth]{planes_of_computing}
	\label{fig:7}
	\caption{Planes of NFV and MEC Architectures}
\end{figure}

These technologies can be grouped into 4 different planes based on their functionalities.

\begin{enumerate}
    \item Data Plane
	\begin{itemize}
	    \item Follows the forwarding logic and responsible for physically moving the packets in the network.
        \end{itemize}
    \item Control Plane
	\begin{itemize}
	    \item Computes the forwarding logic as required by the other planes and programs them to the Data plane and monitors the data plane.
        \end{itemize}
    \item Management Plane
	\begin{itemize}
	    \item Responsible for orchestration and management of resources as required by the services.
	\end{itemize}
    \item Application Plane
	\begin{itemize}
	    \item Makes use of the services through Service Abstraction and APIs from Management to deliver end-user applications.
	\end{itemize}
\end{enumerate}

NIST Cloud Computing group defines cloud computing \cite{nist} as "a model for enabling ubiquitous, convenient, on-demand network access to a shared pool of configurable computing resources that can be rapidly provisioned and released with minimal management effort or service provider interaction". It also enumerates three service models for cloud providers. 

\begin{enumerate}
    \item Infrastructure As A Service
        The provider provides infrastructure and APIs to the consumer to provision processing, storage and network resources allowing the consumer to run arbitrary applications and services.
    \item Platform as a Service
        The consumer can write applications using the programming languages, tools and libraries provided by the provider.
    \item Software As A Service
        The consumer runs the application provided by the provider using a suitable interface like a web-browser or a specialized program. 
\end{enumerate}

\begin{figure}[h!]
	\centering
    \includegraphics[width=0.9\textwidth]{cloud_service_model}
	\label{fig:8}
    \caption{Service Models for Cloud Computing \protect\cite{cloudservice}}
\end{figure}

\input{./content/nfvi}
\subsection{Virtualization}

A basic layered architecture of the Virtual Machines running application workloads is as below in [Fig. 10] based on the explanation from [3]

\begin{figure}[h!]
    \centering
    \includegraphics[width=0.5\textwidth]{vm_layers}
    \label{fig:figure9}
    \caption{Components of Virtualization}
\end{figure}

In traditional virtualization, also refered to as hardware virtualization, a hypervisor (eg: KVM) emulates the hardware into a guest and isolates it from other KVM instances. An Operating System can be installed and the KVM instance can then be used as an independent system for any application. The interaction between the application running on the Guest Operating system and the host operating system kernel is through the intermediate interface called the Hypercall. The hypervisor needs a virtual switch which it can connect to using tap interfaces to communicate with other Virtual Machines (KVM instances). Libraries such as Libvirt provide interfaces to interact with the KVM instances with ease.

\input{./content/containerization}
\input{./content/virt_vs_containers}
\input{./content/opensource}
\input{./content/iaas_openstack}
\newpage
\subsection{OpenStack at the Edge}

Several open source communities have been formed to realize Edge computing use-cases using OpenStack. Most architectures use a Cloud and an Edge part that need to run synchronously to achieve the objectives. The key challenges in realizing openstack at the edge is that the components of openstack require large memory footprint along with the messaging and database infrastructure that runs as backbone for openstack services. This cannot be afforded when deploying at the Edge devices which have limited resources. The idea has been to optimize the openstack services to fit the requirements of the Edge device constraints. Few examples of open source effors in this area include Akraino, EdgeX Foundry, StarlingX and a few more. The figure~\ref{fig:figure13} shows the representation of openstack services at the centralized DC and the edge devices. Depending on the computational capability of the edge device, it can be hosted as Large, Medium or Small Edge \cite{osedge}.

\begin{figure}[h!]
    \centering
    \includegraphics[width=0.9\textwidth]{openstack_edge}
    \label{fig:figure13}
    \caption{OpenStack Components for Edge Computing}
\end{figure}

The component services/daemons for each OpenStack can also be further customized based on the size of the Edge Device as illustrated in~\ref{fig:figure14} derived from \cite{osedge}.

\begin{figure}[h!]
    \centering
    \includegraphics[width=0.9\textwidth]{openstack_edge_services}
    \label{fig:figure14}
    \caption{OpenStack Services for Edge Computing}
\end{figure}



\subsection{Kubernetes for NFV}
	
\begin{flushleft}
Kubernetes is a relatively new introduction to the NFV world. The network functions that are made to run in a container are referred to as Containerized Network Functions (CNFs). The CNCF (Cloud Native Computing Federation) is constantly striving to enable Kubernetes as an alternative technology to realize NFV Infrastructure.
\end{flushleft}

\begin{figure}
    \centering
    \includegraphics[width=0.5\textwidth]{kubernetes_arch}
    \label{fig:figure16}
    \caption{A block diagram representation of Kubernetes components. Derived from the description in \cite{k8sarch}}
\end{figure}


\subsection{Kubernetes at the Edge}

\begin{flushleft}
The below diagram shows the architecture of Kubeedge project, an initiative to develop Edge platform using Kubernetes. The details about the project and its source can be found in \cite{kedge}.
\end{flushleft}

\begin{figure}[h!]
    \centering
    \includegraphics[width=0.9\textwidth]{k8s_edge}
    \label{fig:figure14}
	\caption{Architecture of the KubeEdge project. From \cite[What is KubeEdge]{kedgedoc}}
\end{figure}

\begin{flushleft}
The architecture as shown in Figure 14 consists of two parts:
\end{flushleft}
\begin{enumerate}
    \item \textbf{The Cloud Part}
	    The components that run in the data center and provides mechanisms for the edge part to synchronize with the cloud part. The cloud part consists of the following components:
        \begin{itemize}
            \item \textit{Edge Controller} - interfaces with the Kubernetes API Server and syncs with the events and updates with the edge core.
            \item \textit{Device Controller} - Updates and syncs device status using device model and device instance. 
            \item \textit{Cloud Hub} - Prepares the platform for communication between the cloud and the edge (websocket, channelQ, messages)
		\end{itemize}
    \item \textbf{The Edge Part}
        The components that run on the edge device and interact with the end-user equipment. It synchronizes with the cloud part for functions like device management and service updates. It comprises the following components:
		\begin{itemize}
            \item \textit{EdgeD} - A daemon that manages the life cycle of the pod/s on the edge node. It proposes to include a CRI as a replacement for the current docker runtime to ensure a lightweight runtime and also provide options to choose from multiple container runtimes.
            \item \textit{EventBus} - Is responsible for sending/receiving messages over MQTT to/from external clients.
            \item \textit{MetaManager} - Responsible to process, store and query messages between the Edgehub and the EdgeD. 
            \item \textit{Edgehub} - Interacts with the Cloud Hub on the cloud part either through web-socket or using QUIC protocol. The functions include reporting device and edged status to the cloud and sync with cloud-side resources.
            \item \textit{DeviceTwin} - Supports Device management through storing and querying device status, health check, message distribution and membership management.
	    \end{itemize}
\end{enumerate}

\input{./content/conclusion}
%
%===============================================================================
% LATEX-Dokument: Literaturverzeichnis
%===============================================================================
%
\newpage
\phantomsection
% Einstellungen f\"{u}r Literaturverzeichnis
\addcontentsline{toc}{section}{\bibname}

\bibliographystyle{IEEEtran}
% argument is your BibTeX string definitions and bibliography database(s)
\bibliography{./literature/reference}
% Nutzung von Bibtex:
% hier den bib-file einbinden
%
% GATHER{bibfile.bib}
% \footnotesize
% \bibliography{bibfile}
% ansonsten: bbl als tex Datei einbinden
 %\input{KTR-Seminar-Literatur.tex}
%===============================================================================
% LATEX-Dokument: Literaturverzeichnis
%===============================================================================
%
\end{document}
