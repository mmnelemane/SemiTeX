\section{Outline and Structure}
This report primarily refers to \cite{taleb17} which offers a survey of technologies leading to 5G realization focusing on Edge computing. This report tries to provide an insight into the requirements and architecture of the Edge Computing and its enabling technologies. At the lowest level, there are two approaches to acheiving the goals of the said architectures - Virtualization and Containerization. The report describes the concepts involved in these two approaches and the related enabling technologies that facilitate the building of the complete system and also provide an evaluation through qualitative and quantitative comparisons. 

In the first section, the paper introduces the concept and benefits of MEC. The second chapter discusses the details of the reference architectures proposed in literature for realizing MEC and the related technology NFV. The third chapter lays out the requirements which must be satisfied for such a system. The fourth chapter is a description of the enabling technologies that intend to meet these requirements and here it also describes Virtualization and Containerization approaches in detail along with the corresponding management technologies. In the fifth chapter, we can see an evaluation of Containerization and Virtualization comparing them on various Qualitative and Quantitative factors as elaborately done in \cite{sharma16}. The sixth chapter discusses the two of the popular open source middleware technologies using Virtualization and Containerization respectively and their approach to NFV and Edge Computing. Finally, the report concludes by presenting the future work and general market trend towards Edge Computing.
